% (This file is included by thesis.tex; you do not latex it by itself.)

\begin{abstract}

% The text of the abstract goes here.  If you need to use a \section
% command you will need to use \section*, \subsection*, etc. so that
% you don't get any numbering.  You probably won't be using any of
% these commands in the abstract anyway.

Recent advances in collecting sequencing data from tumors
is promising for both immediate individual patient treatment and investigation
of cancer mechanisms. A resultant central goal is identifying changes in tumors
that are impactful towards these ends. Here, we develop two tools to
identify impactful changes at different levels. We develop both methods
in the context of gliomas, a common form of brain cancer. Firstly, we develop
and assess a tool for assessing the impact of fusion
genes, a type of common mutation using RNA-sequencing data. We
validate the tool by working with collaborators in The Cancer Genome
Atlas. Secondly, we develop a tool for an overall assessment of patient
outcome by integrating data from diverse sequencing platforms. We
validate this tool using simulation, data from consortiums, and
collaborators at UCSF.



\end{abstract}
