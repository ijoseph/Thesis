\chapter{Properties of analyzed data}

Data from several sources were analyzed for the purposes of
identifying impactful sequencing-accessible changes. In particular,
data from TCGA \cite{_comprehensive_2015}, the Costello Lab at
UCSF\cite{johnson_mutational_2014} and the Cancer Cell Line
Encyclopaedia\cite{barretina_cancer_2012} were used.

\section{The Cancer Genome Atlas}

Data from TCGA LGG were used to assess the fusion transcript
expression estimation tool described in chapter 3. $289$ tumors were
analyzed by the TCGA LGG Analysis Working Group (AWG).

\subsection{Raw molecular information}

\subsubsection{RNA-sequencing}
RNA-sequencing protocol was standardized across TCGA data collection
centers. In particular, total RNA extraction was performed using
$5 \times 10^6$ cells using Ambion\textsuperscript{\textregistered}
Ribopure\textsuperscript{\texttrademark} kit. Reads were sequenced
with Illumina\textsuperscript{\textregistered}
HiSeq\textsuperscript{\texttrademark} sequencers, collecting
$2 \times 50$ base-pair paired-end reads.

\subsubsection{Exome sequencing}
Exome sequencing on tumor tissue was performed using $0.5$ to $3 \mug$
of DNA from tumor and normal blood, respectively. The Agilent\reg
Sure-Select Human All Exon \tm v.20, 44 MB kit was used for exome
region-targeting, and $2 \times 76$ base-pair paired-end sequencing
reads were used, again with Illumina\reg HiSeq\tm sequencers.


\subsubsection{Deep whole-genome sequencing}
This was performed on $21$ of the $289$ samples. $2 \times 101$
base-pair paired-end reads were used on the same sequencing platform
as exome sequencing.

\subsubsection{``Low-pass'' whole-genome sequencing}

This was performed on $52$ of the $289$ samples. Here, $0.5$ to
$0.7\mu g$ of DNA were extracted, then KAPA Biosystems\reg kits were
used for preparation. $2 \times 51$ base-pair pari-end reads were
produced with Illumina\reg HiSeq\tm, producing $5.2 \times$ average
coverage. $39.5$ average Phred quality score was achieved, with a
$96.5\%$ average mapping rate.

\subsection{Clinical information}

To assess the impact of sequencing data relative to various
outcomes and to also correlate various molecular features with known
features, classical clinical information was obtained in addition to
the molecular inforation. In particular, histologic type, age of
diagnosis, race, year of diagnosis, family history of cancer, extent
of tumor resection, tumor location, white matter percentage, and first
presenting symptom were collected from all of the $289$ patients, with
a few omissions. Most notable in terms of fusion expression analysis
was survival time, which correlated molecular tumor type and
relatedly, the existence of fusions of interest.

\subsection{Resulting datatypes and details of processing}

The above molecular information was processed, resulting in several
different high-level genomic features.

\subsubsection{DNA rearrangements}

Several different DNA rearrangements sets were estimated from data;
these were used in order to prioritize resultant FJs.
\begin{description}
\item[genome-wide] Two sets of genome-wide (as opposed to
  exome-limited) DNA rearrangements were produced.

  
  \textbf{deep-sequencing-based}

  
  Here, BamBam\cite{sanborn_double_2013} was used to estimate
  rearrangements based on exome-sequencing and deep whole-genome
  sequencing reads. These reads were aligned and processed using the
  aligners of BWA \cite{li_fast_2009} and Picard \cite{_picard_2016} Broad
  Institute Firehose \cite{_firehose_2016}
  
  \textbf{shallow-sequencing-based}

  As deep sequencing was only available on a minority of samples,
  shallow (``low-pass'') sequencing was also used to call DNA
  rearrangements.

  For this purpose, the CASAVA \cite{_casava_2016} tool was used, along with
  BWA for mapping and BreakDancer\cite{chen_breakdancer_2009} and
  Meerkat\cite{yang_diverse_2013}.




\item[intragenic]

  As more sequencing data was available to support genic region-based
  rearrangemnets, this was used to estimate rearrangemnets that was
  directly applicable to FJs.

  \textbf{MapSplice pipeline} MapSplice\cite{wang_mapsplice:_2010} was
  used to call FJs as part of the general RNA-seq alignment pipeline.


  \textbf{deFuse pipeline} deFuse\cite{mcpherson_defuse:_2011} was
  used by the Haussler lab at UCSC in order to call FJs from
  RNA-sequencing data. As part of the deFuse pipeline,
  TopHat\cite{trapnell_tophat:_2009} was used, along with a support vector machine
  (SVM) classifier to assess the probability of fusion existence based
  on priors collected by a test set.

  Reads were further filtered by Drs. Olena Morozova and Sofie Salama
  at UCSC based on criteria outlined in \ref{filtering}. Beyond
  \ref{filtering}, specific FJs bisecting (1) protein kinases, which
  have a known role in higher-grade gliomas and (2) genes
  significantly mutated by smaller mutations were prioritized for
  validation.

  \begin{table}
    \begin{center}
      \begin{tabular}{|c|c|} \hline \textbf{Criterion} &
        \textbf{Accepted Value} \\ \hline
        split reads & $\geq 3$ \\
        spanning reads - split reads & $\geq 0$\\
        probability from SVM classifier & $ > 0.65$\\
        read through event & \texttt{false}\\
        p-value from supporting reads' production from other
        genomic areas & $\leq 0.1$ \\
        fraction of supporting reads that are repetitive sequence
                                                       & $< 0.78$\\
        number of spanning reads & $> 1$ \\ \hline
      \end{tabular}
    \end{center}
    \caption{Criteria for filtering fusions downstream of
      deFuse} \label{filtering}
  \end{table}

  \textbf{PRADA pipeline}

  PRADA was developed and used by collaborators at the MD Anderson Cancer Center as
  an independent method of assessing fusions to increase
  confidence. Briefly, the pipeline involves RNA-seq read alignment,
  followed by filtering by the following shown in \ref{prada_filtering} 

  \begin{table}
    \begin{center}
      \begin{tabular}{|c|c|} \hline \textbf{Criterion} &
        \textbf{Accepted Value} \\ \hline
        exon-connecting reads & $ > 1$ \\
        spanning reads & $> 2$\\
        \texttt{BLASTN}\cite{altschul_basic_1990}-computed inter-constituent sequence
                                                       similarity & $
                                                                    <0.01$\\ \hline
      \end{tabular}
    \end{center}
    \caption{Criteria for filtering fusions as part of PRADA pipeline} \label{prada_filtering}
  \end{table}   
\end{description}


\subsubsection{Gene expression}

Gene expression data was used to validate and prioritize fusions for
further fusion-transcript-gnostic expression estimation. Gene
expression resulted from RNA-sequencing data and was produced based on
the following pipeline:

\begin{enumerate}
\item MapSplice for mapping \cite{wang_mapsplice:_2010}
  
\item RSEM\cite{li_rsem:_2011} for expression estimation
  
\item Filtering to genes expressed in $\geq 70\%$ of samples
  
\item $\log_2$ transformation  
\item Median-centering across samples
\end{enumerate}

\section{UCSF Department of Neurosurgery glioma samples}

Roughly one hundred samples ($86$ at time of writing) were collected by the UCSF department of
neurosurgery and analysed further by the Costello lab. Roughly fifty
($48$ at time of writing) of these are relevant to the prediction problem that I formulate in
chapter $4$, based on arising from patients with adjuvant Temozolomide
chemotherapy. 

\subsection{Raw molecular information}
\subsubsection{Matched normal samples} \label{matched_normal}
For roughly $40$ of the patients, samples from circulating blood were
taken and sequenced via a targeted exome sequencing platform with
Illumina$\reg$ reads.

\subsubsection{Primary tumor samples}
Primary tumor samples were tumor samples collected from the first
surgery for LGG patients. This was used in later analyses for the
purpose of prognostication regarding outcomes following primary
surgery. 

On several patients, several tumor sections from one tumor were
sampled in order to assess intratumoral heterogeneity. On a minority
of these, sections were chosen to maximize variance of metabolic
positron-emission tomography (PET)-based imaging as a proxy to maximize genetic variance of sections.

In detail, PET assessed the presence of choline, which is a marker
for cell membranes. Cell membrane existence is a marker of cell
turnover and therefore a marker fo tumor growth.

Choline is related to lipid isocitrate dehydrogenase 1 (IDH1)
operation metabolism, which is known to be altered in some gliomas, so
is particularly relevant for glioma metabolic assessment. 

\begin{description}

\item[Exome sequencing] For $27$ patients, exome sequencing was
  performed on at least one section of tumor tissue using identical
  protocol to \ref{matched_normal}.

  
\item[RNA-sequencing] For around ten patients, RNA-sequencing data was
  collected using Illumina\reg paired-end read sequencing
  protocols. RNA-seq was not performed on more patients due to
  degradation of RNA.

  
\item[Human methylation 450k array] For roughly $20$ patients, the
  Infinium\reg HumanMethylation450 BeadChip Kit\tm (450k) was used to assess
  CpG methylation at around $500,000$ probe sites.
  
\end{description}


\subsubsection{Secondary tumor samples}

Tumor samples were collected from secondary and following surgeries if
available; the same datatypes were available as was available on
primary tumor samples.

\subsection{Resulting datatypes and processing details}
\subsubsection{Germline variants}
Germline variants were assessed from exome sequencing of the normal
tumor blood samples and processed via the GATK Unified genotype
\cite{mckenna_genome_2010}. Briefly, this method works by aligning reads and
producing variants at sites differing from an input reference, and
results in a list of called variants for every sample on a base-pair
resolution.


\subsubsection{DNA methylation}
Raw probe intensity values were processed by
\texttt{Methylumi}\cite{davis_package_2013}. In detail, probes were
normalized by removing probes with high p-values for detection tests,
then background-corrected. Sex chromosomes were filtered, as were
probes mapping to SNPs and probes mapping to multiple genomic
locatons.

This resulted in per-sample vectors of $\beta$ values representing average methylation level
of CpG site across a sample.

\subsubsection{DNA copy number}

Genomic copy number segments are called from exome-sequencing
information due to newly developed pipelines, and validated with
intensity information from 450k.

This results in segments of floating point-numbers representing copy
number, averaged across cells in a sample. To further process, we took
floating point numbers more than $0.5$ away from $2$.

\subsubsection{Somatic mutations}

Somatic mutations were called based on combining normal and tumor
exome-sequencing information, using GATK. This results
each variant of one or more nucleotides on a base-pair resolution
between tumor and normal sequencing information.

There were a group of samples where the number of mutations was more than $10 \times$
higher, on average, than the other group of samples; these samples
were considered to having undergone \textit{hypermutation} (HM). 

\subsubsection{RNA-sequencing}

RNA-seq data was analyzed via a pipeline using
Kallisto\cite{bray_near-optimal_2016} to generate expresion
estimates. 

\subsection{Clinical information}

A wealth of clinical information is available on the patients. This
was used to select an appropriate molecular phenotype that accorded
well with clinical outcomes. HM was selected as being
related to malignant transformation and relatedly, decreased survival
time.

Clinical information was also used as a predictor for the HM. See
\ref{avail_clin} for details.


  \begin{table}
    \begin{center}
      \begin{tabular}{|c|c|} \hline \textbf{Type} &
        \textbf{Unit} \\ \hline
        age & binned into $5$-year intervals \\
        tumor location & $\in$ \{frontal, parietal\} \\
        extent of resection & $\in$ \{gross total, subtotal\} \\
        length of TMZ cycles & months \\
        time between surgery and TMZ administration & months \\
        time between TMZ administration and recurrence & months \\
        gender & $\in$ \{male, female, neither\}\\ \hline 
      \end{tabular}
    \end{center}
    \caption{Clinical information available} \label{avail_clin}
  \end{table}


\section{Cancer Genome Project}

The cancer genome project is a major consortium effort led by the Wellcome
Trust Sanger Institute to assess the sensitivity of various cancer
types (\ref{cgp_types}) to existing pharmaceutical therapies.


\subsection{Cancer types}

At least one cell line each of from $29$ cancer types were used (\ref{cgp_types}).
resulting in $1,001$ different samples.

\begin{table}
\begin{center}
\begin{tabular}{|c|} \hline
\textbf{Names} \\ \hline
Adrenocortical carcinoma\\
Acute lymphoblastic leukemia\\
Bladder Urothelial Carcinoma\\
Breast invasive carcinoma\\
Cervical squamous cell carcinoma and endocervical adenocarcinoma\\
Chronic Lymphocytic Leukemia\\
Colon adenocarcinoma and Rectum adenocarcinoma\\
Lymphoid Neoplasm Diffuse Large B-cell Lymphoma\\
Esophageal carcinoma\\
Glioblastoma multiforme\\
Head and Neck squamous cell carcinoma\\
Kidney renal clear cell carcinoma\\
Acute Myeloid Leukemia\\
Chronic Myelogenous Leukemia\\
Brain Lower Grade Glioma\\
Liver hepatocellular carcinoma\\
Lung adenocarcinoma\\
Lung squamous cell carcinoma\\
Medulloblastoma\\
Mesothelioma\\
Multiple Myeloma\\
Neuroblastoma\\
Ovarian serous cystadenocarcinoma\\
Pancreatic adenocarcinoma\\
Prostate adenocarcinoma\\
Small Cell Lung Cancer\\
Skin Cutaneous Melanoma\\
Stomach adenocarcinoma\\
Thyroid carcinoma\\
Uterine Corpus Endometrial Carcinoma\\ \hline
\end{tabular}
\end{center}
\caption{Cancer Genome Project Tumor Types for Derived Cell Lines} \label{cgp_types}
\end{table}

\subsection{Compounds information}

\subsubsection{Raw compound details}

A total of 265 compounds were tested, consisting both of drugs under
investigation and established therapeutics. The compounds can be
considered in several categories based on their molecular targets (\ref{cmpnds}) if
available. Broadly, compounds were categorized as cytotoxic or
targeted.

\begin{table}
\begin{center}
\begin{tabular}{|c|} \hline
\textbf{Categories} \\ \hline
ABL signaling\\
DNA replication\\
EGFR signaling\\
ERK MAPK signaling\\
Genome integrity\\
IGFR signaling\\
JNK and p38 signaling\\
PI3K signaling\\
RTK signaling\\
TOR signaling\\
WNT signaling\\
apoptosis regulation\\
cell cycle\\
chromain  histone acetylation\\
chromatin  histone methylation\\
chromatin  other\\
cytoskeleton\\
metabolism\\
mitosis\\
other\\
p53 pathway \\ \hline
\end{tabular}
\end{center}
\caption{Cancer Genome Project Compound Categories} \label{cmpnds}
\end{table}

We focus on analysis of the DNA replication and Cytoskeleton-based
categories, both of which are the only cytotoxic classes tested; the
rationale for this was that these might be more similar to the
phenotype of interest in UCSF datasets.

\subsubsection{Resulting sensitivity estimates}

For each combination of compound and cell line, a call of
\texttt{resistant} or \texttt{sensitive} was made based on the
half maximal inhibitory concentration $\text{IC}_{50}$ values.

A heuristic outlier procedure with the following steps was used for
each compound:

\begin{description}
\item[1. upsampling]
  For each cell line, a normal distribution was fit with mean equal to
  the average $\text{IC}_{50}$ over the different samples, and
  standard deviation equivalent to a $68\%$ estimated confidence
  interval. $1,000$ points were then sampled from the fit normal
  distribution.

  
\item[2. density estimation]
  A density was estimated from the $1,000$ points using kernel
  density estimation with normal kernels.

  
\item[3. modeling resistant population]
  Based on the shape of the resultant density estimation, a resistant population of cell lines, for each compound, were
  identified in an automated fashion. Briefly, a strong prior that the
  majority of cell lines would be resistant was a major assumption
  that went into this; the method attempted to find bimodal structure
  consisting of a mixture of normal distributions; it then assigned
  the node with the higher $\text{IC}_{50}$ as the resistant
  population. 

  
\item[4. calling threshold based on cumulative distribution function]
  Finally, a threshold was called using an empirical cumulative distribution on the estimated
  sensitive population so as to include most of these. 
\end{description}



\subsection{Genomic information}

Broadly, three different molecular assays were performed on each
cell-line-derived sample. 

\subsubsection{Raw genomic information}

\begin{description}
\item[exome sequencing]
  A 64-gene panel was sequenced using capillary (Sanger\reg)
  sequencing.

  
\item[copy-number array]
  Affymetrix\reg SNP6.0\tm microarrays were used to assess copy
  number. 

  
\item[expression array]
  Affymetrix\reg HT-U133A\reg microarrays were used to gather gene
  expression information based on RNA transcript abundance-based
  annealing to predefined probes.  
\end{description}


\subsubsection{Resulting datatypes and processing details}

\begin{description}
\item[variants]
  Variants were called within
  the 64-gene panel as compared to reference. As matched normals were
  not obtained, variants were a mixture of both somatic mutations and
  germline variants, likely dominated by somatic mutations.
\item[copy number]
  Copy number information was assessed from the copy-number array
  using Affymetrix\reg processing tools.

  
\item[gene expression]
  Array-based expression estimates were obtained for the $14,500$
  genes available on the platform panel.

  
\item[gene fusions]
  Fusions were called based on the Sanger\reg sequencing. 

\end{description}










