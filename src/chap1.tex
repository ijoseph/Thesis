\chapter{Goal of collecting genomic data from cancers}

\section{Context}
In the start of the second millennium, scientific inquiry as a whole
has remained relatively absent from political discourse in the United
States and relatedly,
the collective consciousness of the nation. Part of this may be due to
the ending of the Cold War around 20 years prior, which removed the
impetus for international scientific competition; another part may be
due to the sufficient progress science has made in many fields,
leading the way to technologies and engineering to take the forefront
and to translate discoveries into general comfort.

Human health remains an important area, although even here science seems to
have sufficiently addressed most realms – infectious diseases and
most non-infectious diseases are now much less deadly than they were
in the recent past. To those with sufficient means in the United
States and most western countries, developments based on scientific
inquiry serve them well.

Arguably the biggest exception to this is cancer; in the last 100
years, cancer mortality as a whole has only very slightly
decreased. One reason for this may be the mechanistic uniqueness of
the disease in every individual.

Encouragingly, the sequencing of the human genome in 2001\cite{lander_initial_2001} and the
subsequent decreasing cost of genomic sequencing methods has yielded
some promise in cancer treatment possibilities. In particular,
machines that can gather sequencing information from tumors (“genomic
data”) may be able to be used for an increasing amount of the
population, financially, and these might be able to lead to the
appropriate handling of uniqueness to decrease mortality.

Additionally, the study of cancer is central to the study of molecular
biology as a whole; the transformation of a collection of somatic
cells into tumor tissue can be seen as a betrayal of the fundamental
contract underlying the constitution of a multicellular organism, in
that several semi-autonomous components cooperate towards a greater
goal of organismal life and reproduction. All biological systems of
organization involved in this transformation, and therefore the study
of cancer is equivalently the study of human disease as a whole. This
means the study of cancer has global reach, if not imminently,
ultimately.


In 2015, United States’ President Barack Obama announced the Precision
Medicine initiative during his State of the Union address. During the
same address in 2016, he announced the National Cancer Moonshot
Initiative. Both were rightfully meant to focus attention on this
problem.

In this dissertation, I make use of two large-scale
multi-institutional consortium-based efforts and one smaller-scale
intra-institutional effort to idnetify impactful genomic alternations
using novel tools and approaches. First, I outline the overall goals
of the collection of this data, which are twofold:


\section{Investigation}
	To successfully use genomics data for tumor prognosis and treatment
        decisions on a personal level, much still needs to be
       collected, as the relationship between collectable genomics data and 
       above outcomes has not been well established. There  are several ongoing and recently completed large-scale efforts to collect this data in order to achieve this. These can be
        seen as investigations towards understanding specific areas of
        unknown: the mechanisms of (1) cancer onset (\textit{oncogenesis}),
        (2) \textit{progression}, and (3) \textit{metastasis} (spread to regions of
        the body distant from the original tumor site).


        \subsection{Identification of molecular subtypes}

	One area of unknown is the existence of molecular subtypes of
        tumors – groups of tumors based on the molecular aberrations
        they share. This is in contrast with the classical tissue
        definition of tumors, which is based on the type of tissue
        from which they originated. The Cancer Genome Atlas (TCGA)\cite{mclendon_comprehensive_2008}, a
        major consortium which gathered 11,000 tumor samples from 33
        classically-defined cancer types, aims to help assess this
        difference. I later describe identification of impactful
        features from one tumor type contained in TCGA. 
        
        Analyses under the TCGA, which gather sequencing data and compare it
        across tumor types, have made findings suggesting that
        a significant fraction tumors defined classically would be
        better assessed based on their molecular similarity with a
        non-classically equivalent tumor.

        One area of success in molecular subtype definition has been
        gliomas. Gliomas are one of the most common forms of brain
        cancer, comprising 30\% of brain and central nervous system
        and 80\% of all malignant brain tumors. Gliomas are
        classically classified by World Health Organization (WHO)
        grade (II through IV for adults), and the glial subtype that
        the tumor cells most resemble based on examining tumor tissue
        from biopsies under the microscope (histopathology):
        oligodendroglioma, astrocytoma, or oligoastrocytoma for tumors
        appearing similar to oligodendrocytes, astrocytes, or both,
        respectively. 
        

        Somewhat expectedly based on prior related research which suggested
        yet was not able to provide sufficient evidence for a definitive
        conclusion, TCGA researchers studying sequencing information from over
        300 lower-grade glioma (LGG) tumors identified three molecular
        subtypes of the tumors based on sequencing information. LGG tumors
        were defined based on histopathologists’ assessments of tumor tissue
        as being indicative of grade II of any histopathology.


        This led to finding consistent subgroups of patterns across
        all types of  sequencing information. Importantly, these
        subgroups were found to correlate more closely with important
        patient outcomes than the classical classifications (grade and
        glial subtype resembled). Interestingly, researchers were also
        able to find that subtypes, while evidenced by much sequencing
        information across the entire genome, were well defined by a
        very small number nuclear changes as well, suggesting possible
        targets for therapies. Researchers also identified that one
        molecular subtype appeared very similar to tumors whose
        classical classification would have indicated a higher grade,
        and these tumors had similarly poor prognosis as these
        higher-grade tumors.


        

        In addition to finding subtypes within one specific classical tumor
        diagnosis category and comparing those subtypes with other classical
        categories in an ad-hoc manner, TCGA has also implemented the search
        for subtypes with so-called Pan-cancer analysis\cite{weinstein_cancer_2013} studies. Based on
        clustering, researchers assessed classes of tumors in a classical
        category-agnostic fashion. This is consistent with the molecular
        understandings of (initiation and onset of tumor tissue
        (oncogenesis). In particular, the same general mutational patterns
        involving the same classes of genes are likely to have the ability to
        initiate tumor-like properties across classical tumor
        types.

        In the context of cancer genomics, mutations are defined as changes in the DNA of a
        tumor that are not also present in the DNA of non-tumor
        cells. These are sometimes termed ``somatic mutations,'' to
        underscore their occurrence within the tumor tissue of the
        individual in which they are detected, and are contrasted with
        ``germline mutations.


        The major finding here in terms of subtypes was the stratification of
        classical tumor categories on a spectrum from tumors with a high
        mutational burden (many small aberrations in genomic DNA present in
        tumor tissue as compared to normal tissue derived from blood) to those
        with a high copy number change burden (many large-scale aberrations in
        genomic DNA). Interestingly, no classical category appeared to have
        high amounts of both type of mutation; this may point to similarities
        in oncogenesis mechanisms in tissues that are similar in terms of
        mutational burden type, and two general oncogenic tumor
        categories.
        
        \subsubsection{Assessing clinical utility of molecular subtypes}

        A further area of investigation is whether subtypes identified
        based on consistent molecular patterns have clinical utility.

        
        One promising result is in gliomas, where the finding that molecular subtypes were more
        accurate at distinguishing clinical outcomes can now be used by
        clinicians to treat patients with the privilege of having had
        their genome sequenced; the WHO is in the process of updating
        the standard of care for LGG patients, which will formalize
        this new ability to treat patients more effectively. In
        particular, those with a prognosis similar to higher grade
      tumors based on molecular subtyping can be treated
        accordingly; this may involve more aggressive use of
        chemotherapy and/or radiotherapy.

        Towards this end, the Cancer Cell Line
        Encyclopaedia (CCLE)\cite{barretina_cancer_2012} and the Cancer
        Genome Project (CGP) \cite{garnett_systematic_2012}are testing a variety
        of new targeted therapies \textit{in vitro} in a relatively knowledge-agnostic
        approach. I use data from both consortiums towards this goal,
        as well. Targeted therapies are pharmaceutical agents that
        interrupt the activity of a particular molecular process,
        typically by interfering with a protein. As part of the
        rational development of cancer therapies, they are typically
        chosen in a strategic way to interrupt key pathways that are
        important in particular types. The molecular similarities
        between tumors of different types justifies the relatively
        unbiased approaches used by both studies, which contrasts with
        a one-tissue-type, one-drug approach.
        
        In particular, for  example, drugs targeting the epidermal growth factor receptor
        (EGF) pathway have been shown to be effective against multiple
        different tumor tissues of origin
        \cite{barretina_cancer_2012}.


        \subsection{Assessing relative impact of genetic and
          epigeneitc changes}

        A second area of open investigation is the role of epigenetics
        in oncogenesis, tumor progression, and
        metastasis. Epigenetic changes consist of non-base-pair
        DNA changes to the genome based on modification of the
        genome's surroundings; in particular, chromatin can be opened
        or closed via a myraid of mechanisms, such as histone
       modifications and DNA methylation\cite{kouzarides_chromatin_2007}.

        These have an unknown degree of impact;
        it is not known, for example, whether such changes
        (``epimutations'') are sufficient for oncogenesis
        \cite{nagarajan_recurrent_2014} \cite{feinberg_epigenetic_2006}. 

        Towards that end, Dr. Joseph Costello, Ph.D. is spearheading a
        University of California, San Francisco (UCSF)-based approach to
        gather both genetic and epigenetic data on LGG tumors. He has
        found already DNA methylation-based epigenetic changes
        correspond closely with alternations in DNA base
        pairs\cite{johnson_mutational_2014}. 

        \subsection{Identification of driver mutations}

        Another general goal of collecting genomic data from tumors,
        and one of the first goals, has been the identification of
        previously unobserved  ``driver'' mutations. Driver mutations are DNA changes that
        are  ``causally implicated
        in oncogenesis''\cite{stratton_cancer_2009}. Driver mutations
        are typically validated via \textit{in vitro} and \textit{in vivo}
        experiments. The number of possible driver mutations is as
        endless as the number of possible mutations, so therefore
        candidate driver mutations need to be established in order to
        narrow down the search space of possible drivers. 

        One common prioritzation technique is to identify somatic
        mutations that occur more often than what might be expected
        based on chance (``recurrent mutations''). This technique
        requires statistical power, necessitating, in turn, the
        collection of many tumor
        samples\cite{leiserson_pan-cancer_2015}. In particular, power
        is required to identify candidate driver mutations with either
        low-impact or rare occurence. Many of these are believed to
        exist, based on similar properties of variants detected during
        the conceptually similar genome-wide association studies
        (GWAS), which search for germline variants associated with a
        particular phenotype, typically disease-related. 
        
        Further prioritization may be obtained based on somatic
        mutation annotations, which may be based on prior knowledge of
        a particular mutation's genomic context and its function. In particular,
        intergenic status, knowledge of domain structure, or knowledge
        of regulatory regions might be used to prioritize one
        candidate driver mutation for testing over another. Most
        somatic mutations are point mutations\cite{lawrence_mutational_2013}, although 
        many will involve normally disparate genomic regions, which
        are used in a similar fashion.

        Many tools have been developed to harness this data, and many
        have been applied to the specific datasets I explore later
        (TCGA,CGP, UCSF), including MuTect
        \cite{cibulskis_sensitive_2013}.


        \subsection{Assessing relative impact of germline and somatic variants}

        Some cancers are known to have a strong germline genetic component in
        terms of incidence risk \cite{stacey_germline_2011}; this is especially true of
        childhood tumors. However, in general, it is unknown for all
        cancer types how much of an effect germline variants, rather
        than somatic mutations, have in terms of oncogenesis,
        tumor progression, and metastases. This question can be probed
        with large scale data        
        involving matched normals, which allows the separation of
        germline variants from somatic mutations and also the tracking
        of tumor outcomes.

        This is one of the aims of collecting sequencing data from glioma patients in the
        Costello lab; importantly, follow-up information is present
        for patients, which is nontrivial due to the high mean
        survival time of such patients (on the scale of 10 years,
        depending on molecular subtype).

        \subsection{Molecular mechanisms explaining variance in treatment
          response}

        In some cancer types, there exist a wide variance in how
        patients respond to identical treatment and identical cancers
        (modulo to classical clinical covariates). The molecular
        reasons behind this are currently unknown in most cases.

        \subsubsection{Gliomas}

        This is especially true gliomas, and uncovering this is a
        major aim for collecting surgical tissue by the Costello lab.

        In particular, LGG patients of subtype astrocytoma (identified
        based on histopathology) vary in their response to current
        chemotherapeutic agents.

        Typically, LGG patients are treated by (1) surgical resection
        of the tumor, and (2) ``adjuvant'' chemotherapy. Chemotherapy
        is typically temozolomide (TMZ), although may also be
        procarbazine-lomustine-vincristine (PCV). The purpose of adjuvant chemotherapy is to kill remaining tumor tissue
        that was not extracted during surgery, which is common due to
        the difficulties of surgery in the context of vital brain
        areas (eloquent regions) that cannot be easily removed or removed at all without
        substantially decreasing the patient's quality of life.

        LGG has an almost 100\% recurrence rate; surgical tissue
        from the recurrent tumor (second surgery) is also collected by the Costello Lab
        for further study. Some patients that receive adjuvant TMZ have been found to
       progress to a higher grade of glioma
       \cite{johnson_mutational_2014}, which is termed undergoing a
       \textit{malignant transformation} (MT).

       Towards the end of understanding why some patients undergo MT
       whereas others don't, the Costello lab is looking for clues in
       the genomic information collected on tumor tissue collected
       during both sugeries.

       One lead is a large fraction TMZ-treated patients that undergo MT 
       have a drastically higher rate of DNA mutations than those that
       do not undergo MT, termed \textit{hypermutation}. This suggests
       the existence of one molecular pathway relating to the
       deviating response to TMZ in these patients involving
       hypermutation, possibly related to the inactivation of
       particular DNA mismatch repair genes in these tumors.

       An open question, then, is if this negative response (MT) can
       be predicted based on surgical tissue collected during primary
       tumors; this is one I attempt to investigate, and describe
       later in this document.

       \section{Making individualized treatment decisions (``precision
         medicine'')}

       A complementary and ultimate goal of collecting genomic data
       from tumor samples is to directly influence the course of an
       individual's cancer treatment regimen. This is motivated by the
       notion that cancer genomics can collect data that can
       distinguish previously indistinguishable cases in a way that
       allows for their more successful treatment.

    Methods for treatment of a particular tumor are based
       on \textit{standards of care}, which are roughly
       legally-defined and clinically-implemented notions of treatment
      regimens for a particular diagnosis. These vary a substantial
      amount between treatment centers, may or may not be formally
      established legally, and change due to new medical findings
      being published in medical journals. They may also be set
      formally by global organizations, such as WHO
      \cite{moffett_standard_2011}.       

       \subsection{Non-genomic methods}
      Many standard methods for cancer treatment do not rely on the
      collection of genomic data from a tumor.
      
      \textit{clinical covariates} are non-tumor, non-genomic attributes such as age and gender
      which may affect which treatment regimen is proscribed for a
      particular patient according to a standard of care. For example,
      a patient of advanced age with a new prostate cancer may not be
      treated, as it will be assumed the patient will die with, rather
      than of, the tumor; a younger patient, however, may be treated
      with surgery and/or cytotoxic chemotherapy.

      Treatment decision methods involving properties of the tumor(s)
      themselves can be divided into \textit{invasive} and
      \textit{noninvasive} procedures.

      \textit{noninvasive} procedures don't involve breaking the skin,
      which for tumor inspection means imaging techniques. For
      gliomas, magnetic resonance encephalopathy (MRI) is used in
      order to assess tumor grade, stage, and possible recurrence, which in turn influences
      treatment decisions. Positron emission tomography,
      electroencephalography, and other similar imaging methods are
      also used for various tumor types as well.These techniques
      generally involve analysis of images to assess tumor location,
      size. Molecular properties can also be inferred through the use
      of specific imaging able to detect various metabolites which may
      be present in a tumor. 

      \textit{invasive} procedures involve breaking the skin. All of
      these are typically considered to be biopses, and involve
      collecting tumor tissue for physical examination. A typical
      method by which biopsy tissue can be examined is
      \textit{histopathology}, which
      involves examining a slice of tumor tissue under a microscope
      after staining procedures. This is done in gliomas in order to
      distinguish grades II, III, and IV based on the presence of necrotic
      tissue, cellularity, and other visible properties.

      \subsection{Genomic methods}

      Increasingly, genomics data is being used clinically and
      therefore collected for the purpose of making treatment
      decisions on an individualized basis.

      \subsubsection{Gene expression}

      Genomic approaches for treatment prognostication and
      stratification have focused on gene expression, often estimated through the use of microarray
      or RNA-sequencing technologies which measure the abundance of
      RNA in cells through sequencing. The successful discovery of a 231-gene expression signature in 2002 related to breast
      cancer prognosis\cite{van_t_veer_gene_2002} spurred the search for signatures in various
      tumor types for various outcomes. There are currently several genomic signatures from various
      tumor types that allow for patient stratification. Several
      metrics of interest exist in order to define \textit{impactful}
     from the context of making treatment decisions. Essentially, all
     are a method prognostication of patient outcome, conditioned on
     specific treatment decisions:
      (1) survival time, (2) susceptibility to treatment with a
      specific drug, (3) grading/ malignancy classification of a
      tumor, and (4) susceptibility to specific events. Events of
      interest include (4.1) tumor progression, (4.2) tumor
      metastasis, which is closely related to (4.3) tumor
      invasiveness, (4.4) functional status of a particular
      molecular pathway within a tumor (related to (2)), and (4.5)
      likelihood of recurrence subsequent to tumor resectilevel
      
      In particular, several of these are currently used in the
      clinic for prognostication, including
      MammaPrint\textsuperscript{\textregistered} 70-gene signature and 
      Genomic Health's OncotypeDX\textsuperscript{\textregistered}
      21-gene signature for
      breast cancer,
      and Veracyte's Affirma\textsuperscript{\textregistered} for
      thyroid cancer. These may be used to decide whether to provide
      chemotherapy or other further follow-up treatment.

      \subsubsection{Somatic mutations}
      Somatic mutations are also increasingly used in the
      clinic. Specific tests exist for signatures involving a few
      genes such that they can be queried a patient-level without the use of
      high-throughput sequencing. Common options include polymerase
      chain reaction (PCR) or antibody stain of collected
      tissues. Also, companies such as Foundation Genomics
      \textsuperscript{\textregistered} perform large-scale targeted
      sequencing for a handfull of genes known to be generally
      impactful or targetable by specific therapies. Specific cancer
      tests include avian erythroblastic leukemia viral oncogene
      homolog 2 (ERBB2) for breast cancer via AviaraDx \textsuperscript{\textregistered}.

      In gliomas, both of these methods are used; an antibody stain for Isocitrate Dehydrogenase 1
      (IDH1) are being used in some clinics for prognostic tests,
      including UCSF. In particular, IDH-wildtype gliomas have a
      drastically lower survival time.
      This is a proximal use of the tissue that is
      further analyzed by sequencing and by the Costello lab. 

      \subsubsection{DNA methylation}

      DNA methylation-based collection is relatively new for the use
      of direct patient stratification; signatures exist for Acute
      Myeloid Leukemia that relate to overall survival
      \cite{figueroa_dna_2010}, but are not currently used
      clinically. 

      Gliomas are at the forefront of this, where the
      status of the methylation of the promotoer of the
      O\textsuperscript{6}-Guanine Methyltransferase (MGMT) gene is
     queried by antibody, which has found to be prognostic. In particular, lack of
      methylation here is associated with high MGMT activity and
      therefore low chemotherapeutic benefit, so is often a major
      component in treatment decisions \cite{rapkins_mgmt_2015}. This
      was a proximal use of tumor tissue further analyzed by the
      Costello lab and myself later in this document.


      \subsubsection{Germline variants}
      Finally, properties of patients' germlines are occasionally used
      for prognostication. In breast cancer, the appropriately titled
      breast cancer 1 gene (BRCA1) and breast cancer 2 gene (BRCA2)
      are queried for prognosis, as there are known heritable
      deactivating mutations in these tumor-suppressor genes. In
      gliomas, there is a known heritability of MGMT-methylation
      response, although this is not currently linked strongly enough
      to germline variants in order to be used clinically.

      \subsection{Integrated sequencing methods}

      Integration of features are not currently used for
      stratification, although multi-type signatures have been
      collected by CGP relating to response to targeted
      therapies, as outlined above. 
      

      



  

      

      

      

      





       
       


        

        
        

        


        