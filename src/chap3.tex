\chapter{Identifying impactful fusion genes: fusion expression
  analysis}


\section{Motivation}
Acquisition of increasing numbers of somatic mutations on a rapid
timescale and in a heterogeneous fashion across different cells within
a tumor or pre-tumor tissue, is widespread and occurs in most cancer
types. Resulting mutations are sufficient to induce tumorigenesis and
also to drive tumor progression and metastasis. This is done
classically through the activation of proto-oncogenes and through the
deactivation of tumor suppressor genes.


There are several specific pathways by which mutations can be induced
and several corresponding classes of mutations\cite{stratton_cancer_2009}. One class is
large-scale genomic rearrangement, which involves the breaking and
possible rejoining of multiple-megabase sections of DNA.


When large genomic regions dislocate, they may often rejoin in a
predictable fashion to regions either on other chromosomes or within
the same chromosome (translocations).  If there are genic regions
spanning the genomic breakpoints, RNA messages may be transcribed that
contain two genes from two constituent chromosomes. Such messages, if
translated, become fusion genes (FGs).


Due to fragile regions in DNA, specific FGs may be formed to a
significant extent in certain tissues during
tumorigenesis\cite{yunis_constitutive_1984}. In chronic myeloid
leukemia (CML), a fusion between breakpoint cluster region (BCR) and
Abelson murine leukemia (ABL) virus genes leads to the recurrent
BCR-ABL fusion which is present in a large percentage of CML
patients. This fusion is known to have potential to transform normal
cells into cells with tumor characteristics, and has been successfully
targeted by Imatinib, one of the world’s first successful targeted
therapies in terms of extending patient overall survival time. Several
similar examples have been discovered; identifying FGs is thus of
primary interest \cite{chin_cancer_2011}. 

One goal of major consortium efforts such as The Cancer Genome Atlas
(TCGA)\cite{weinstein_cancer_2013} and The Cancer Cell Line
Encyclopedia (CCLE)\cite{barretina_cancer_2012} is detecting recurrent
fusion genes from different caner types based on sequencing
information. Recently, for example, recurrent receptor tyrosine kinase
fusions were detected in gliomas\cite{_comprehensive_2015}.

One mechanism whereby fusion genes may have impact on tumor tissue is
via the induced expressional deregulation of the constituents. In
particular, a sequence... #TODO finish


\section{Existing Impact-Assessment Strategies}
\subsection{Detection}
Typically, fusions are detected from RNA-sequencing (RNA-seq) reads,
which are produced due to its relatively low cost compared to whole
genome sequencing and high interpretability. Several computational
tools to detect fusions from RNA-seq reads exist. Most identify fusion
junctions (FJs), which are the breakpoints of the associated
genomic translocations.


Tools identify FJs in three steps: (1) finding chimeric reads (CRs), which is a single read with two portions aligning to two separate genomic locations, (2) aggregating chimeric reads into candidate fusion junctions through realignment-based grouping, and (3) filtering candidate FJs based on heuristic filters. 

While simple in concept, identifying FJs is a problematically error-prone process. Firstly, incorrect read mapping leads to spurious CRs. Incorrect mapping is often a result of repetitive regions in the genome such as germline segmental duplications. Secondly, even if reads are correctly aligned, false positives may be generated by read-through events, where genomically adjacent genes are erroneously transcribed into one RNA message or reverse transcriptase template-switching events/ trans-splicing. These produce low but detectable baseline levels of fusion genes in wild-type cells, but are usually not of interest in cancer sequencing efforts as they are not causally involved in tumorigenesis, cancer progression, or metastasis\cite{gingeras_implications_2009}. 

Thus, candidate FJs must be filtered aggressively by heuristics based on knowledge of the above. One problem is that it’s not clear exactly which heuristics to use; many heuristics based on ignoring FGs in repetitive regions, for example, may filter real fusions\cite{kumar_identifying_2016}. Another is that given a set of heuristic filters, it’s not clear how to best combine the heuristics in a way that’s generalizable to most FG discovery use cases. This is due in part to the wide range of genomic instability that tumors from different tissues have exhibited. A symptom of this is that Machine learning-based classifiers tuned on representative datasets have issues generalizations to new tumor types. The high false-positive rate stymies further discovery and assessment of FGs. 

FJ identification is also a very computationally-intensive process, as every read must be split in a number of ways and positions and matched against all possible regions of the genome through alignment during CR finding steps10. One of the reasons for this is that existing fusion discovery methods use computationally intensive first-generation alignment algorithms. 

\subsection{Expression Quantification}



% We propose an improved method for FG discovery based on pseudoalignment\cite{bray_near-optimal_2016}. This method achieves high specificity based on ?, and is vastly more computationally efficient than previous methods via the use of pseudoalignment, which uses exact-hashing of k-mers. 







